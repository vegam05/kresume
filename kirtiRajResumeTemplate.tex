\documentclass{resume} % Use the custom resume.cls style

\usepackage[left=0.4 in,top=0.4in,right=0.4 in,bottom=0.4in]{geometry} % Document margins
\newcommand{\tab}[1]{\hspace{.2667\textwidth}\rlap{#1}} 
\newcommand{\itab}[1]{\hspace{0em}\rlap{#1}}
\name{Kirti Raj} % Your name
% You can merge both of these into a single line, if you do not have a website.
\address{+91 6206906750 \\ Madhubani, India} 
\address{\href{mailto:work.kirti005@gmail.com}{Email} \\ \href{https://www.linkedin.com/in/vegam005}{LinkedIn} \\ \href{https://vegam05.pythonanywhere.com}{Portfolio Website} \\
\href{https://github.com/vegam05}{GitHub}}   %

\begin{document}

%----------------------------------------------------------------------------------------
%	INTRODUCTION
%----------------------------------------------------------------------------------------

\begin{rSection}{INTRODUCTION}

{Skilled Software Engineer with expertise in C, C++, Java and Python. Passionate about AI and ML, especially deep learning, proficient with platforms like TensorFlow, Keras, PyTorch, and Detectron2. Experienced in developing and deploying innovative machine learning/deep learning models and integrating AI into software applications. }


\end{rSection}
%----------------------------------------------------------------------------------------
%	EDUCATION SECTION
%----------------------------------------------------------------------------------------

\begin{rSection}{Education}

{\bf B.Tech in Computer Science and Engineering}  \hfill {Expected 2026}\\
Sikkim Manipal Institute of Technology, Majitar\\
{8.58 CGPA}

{\bf CBSE Intermediate(Class XII)} \\ Indian Public School, Madhubani \hfill {2022}
\\{85.4\%}

{\bf CBSE Matriculate(Class X)} \\ Indian Public School, Madhubani \hfill {2020}
\\{93.6\%}
%Minor in Linguistics \smallskip \\
%Member of Eta Kappa Nu \\
%Member of Upsilon Pi Epsilon \\


\end{rSection}

%----------------------------------------------------------------------------------------
% TECHINICAL STRENGTHS	
%----------------------------------------------------------------------------------------
\begin{rSection}{COMPETENCIES}

\begin{tabular}{ @{} >{\bfseries}l @{\hspace{6ex}} l }
Programming Languages & C, C++, Java, Python
\\
Frameworks/Libraries & PyTorch, Detectron2, Qt, NumPy, Pandas, Keras, TensorFlow, Scikit-Learn\\
Databases & MySQL, SQLite\\
Development Tools & VS Code, Google Colaboratory, Git, GitHub, Docker, Jupyter, Vim \\
Operating Systems & Debian, Ubuntu, Arch Linux, Windows 11
\end{tabular}\\
\end{rSection}

\begin{rSection}{EXPERIENCE}

\textbf{AI Intern and Team Lead } \hfill  July 2024 - Present\\
Caare \hfill \textit{Remote}
 \begin{itemize}
    \itemsep -3pt {} 
     \item Lead a team of doctors and annotators ensuring on-time quality annotations across diverse datasets.
     \item Identified critical components actively contributing to heat maps in conjunction with the instance segmentation model being used. 
 \end{itemize}


\textbf{Machine Learning Intern} \hfill  June 2024 - July 2024\\
Bharat Intern \hfill \textit{Remote}
 \begin{itemize}
    \itemsep -3pt {} 
     \item Built a movie recommendation system using collaborative filtering.
     \item Developed a ML model for classifying Iris flowers based on their sepal and petal measurements.
 \end{itemize}
 


\end{rSection} 

%----------------------------------------------------------------------------------------
%	WORK EXPERIENCE SECTION
%----------------------------------------------------------------------------------------

\begin{rSection}{PROJECTS}
\vspace{-1.25em}
\item \textbf{PlatePal} {A food identification and recommendation system called \textit{PlatePal} that employs instance segmentation models from Foodvisor to accurately identify various food items presented to it and also suggest dietary recommendations based on user's goals and health conditions using an LLM model.}\href{https://github.com/vegam05/PlatePal}{(Github Repo)}
\item \textbf{Portfolio Website} {Made a personal portfolio website from scratch using Django as backend alongwith HTML5, CSS and JS. Hosted the project using Python Anywhere's hosting services on free tier.}\href{https://github.com/vegam05/portfolio}{(Github Repo)}
\item \textbf{Kaplayer} {Currently developing this media player completely in C++ using Qt framework for fast and responsive GUI that supports a wide array of media codecs. Future scopes of this project includes database integration for online media hosting and access.\href{https://github.com/vegam05/kaplayer}{(Github Repo)}}
\end{rSection} 

%----------------------------------------------------------------------------------------






\end{document}
